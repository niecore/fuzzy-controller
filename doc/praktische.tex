%http://www.daniel-brettschneider.de/allgemein/latex-vorlage-fur-hausarbeiten-oder-abschlussarbeiten
\documentclass[12pt,a4paper,bibliography=totocnumbered,listof=totocnumbered, abstracton]{scrartcl}

\usepackage[autostyle=true,german=quotes]{csquotes}
\usepackage[utf8]{inputenc}
\usepackage{amsmath}
\usepackage{amsfonts}
\usepackage{amssymb}
\usepackage{amsthm}
\usepackage{graphicx}
\usepackage{fancyhdr}
\usepackage{tabularx}
\usepackage{geometry}
\usepackage{setspace}
\usepackage[right]{eurosym}
\usepackage[printonlyused]{acronym}
\usepackage{subfig}
\usepackage{floatflt}
\usepackage[usenames,dvipsnames]{color}
\usepackage{colortbl}
\usepackage{paralist}
\usepackage{array}
\usepackage{titlesec}
\usepackage{parskip}
\usepackage[right]{eurosym}
\usepackage[subfigure,titles]{tocloft}
\usepackage[pdfpagelabels=true]{hyperref}
\usepackage[ngerman]{babel}
\usepackage{booktabs}
\usepackage{listings}
\usepackage{csquotes}
\usepackage{siunitx}
\newtheoremstyle{Umgebung}	% name
{20pt}	% Space above, empty = `usual value'
{20pt} % Space below
{} % Body font
{} % Indent amount (empty = no indent, \parindent = para indent)
{\bfseries} % Thm head font
{} % Punctuation after thm head
{\newline} % Space after thm head: \newline = linebreak
{} % Thm head spec


\theoremstyle{Umgebung}

\lstset{basicstyle=\footnotesize, captionpos=b, breaklines=true, showstringspaces=false, tabsize=2, frame=lines, numbers=left, numberstyle=\tiny, xleftmargin=2em, framexleftmargin=2em}
\makeatletter
\def\l@lstlisting#1#2{\@dottedtocline{1}{0em}{1em}{\hspace{1,5em} Lst. #1}{#2}}
\makeatother
\geometry{a4paper, top=27mm, left=30mm, right=20mm, bottom=35mm, headsep=10mm, footskip=12mm}
\hypersetup{unicode=false, pdftoolbar=true, pdfmenubar=true, pdffitwindow=false, pdfstartview={FitH},
	pdftitle={Ausarbeitung Projektvortrag Fuzzy-Regelung},
	pdfauthor={Joel Bartelheimer, Nico Müller},
	pdfsubject={Ausarbeitung Fuzzy-Reglung},
	pdfcreator={\LaTeX\ with package \flqq hyperref\frqq},
	pdfproducer={pdfTeX \the\pdftexversion.\pdftexrevision},
	pdfkeywords={Ausarbeitung Projektvortrag Fuzzy-Regelung},
	pdfnewwindow=true,
	colorlinks=true,linkcolor=black,citecolor=black,filecolor=magenta,urlcolor=black}
\pdfinfo{/CreationDate (D:20110620133321)}
\begin{document}
\titlespacing{\section}{0pt}{12pt plus 4pt minus 2pt}{-6pt plus 2pt minus 2pt}
% Kopf- und Fusszeile
\renewcommand{\sectionmark}[1]{\markright{#1}}
\renewcommand{\leftmark}{\rightmark}
\pagestyle{fancy}
\lhead{}
\chead{}
\rhead{\thesection\space\contentsname}
\lfoot{Ausarbeitung Projektvortrag Fuzzy-Regelung}
\cfoot{}
\rfoot{ Seite \thepage}
\renewcommand{\headrulewidth}{0.4pt}
\renewcommand{\footrulewidth}{0.4pt}
% Vorspann
\renewcommand{\thesection}{\Roman{section}}
\renewcommand{\theHsection}{\Roman{section}}
\pagenumbering{roman}
% ----------------------------------------------------------------------------------------------------------
% Titelseite
% ----------------------------------------------------------------------------------------------------------
\thispagestyle{empty}
\begin{center}
	\includegraphics[width=5cm]{img/thm2.png}\\
	\vspace*{2cm}
	\Large
	\textbf{Fachbereich}\\
	\textbf{Mathematik, Naturwissenschaften und Informatik }\\
	\vspace*{2cm}
			\Huge
	\textbf{Ausarbeitung Projektvortrag Fuzzy-Regelung}\\
	\vspace*{1.5cm}
		\small
		\textbf{Im Rahmen der Veranstaltung:}\\
		\Large
	\textbf{Praktikum Künstliche Intelligenz(CS5330)}\\
	\vspace*{2cm}
	

	\normalsize
	\newcolumntype{x}[1]{>{\raggedleft\arraybackslash\hspace{0pt}}p{#1}}
	\begin{tabular}{x{7.5cm}x{7.5cm}}
		\rule{0mm}{5ex}\textbf{Autoren:} 
		\newline 
		\newline Joel Bartelheimer
		\newline joel.bartelheimer@mni.thm.de
		\newline 
		\newline Nico Müller
		\newline nico.mueller@mni.thm.de
		\newline
		& 
		\rule{0mm}{5ex}\textbf{Eingereicht bei:} 
		\newline
		\newline  Prof. Dr. Wolfgang Henrich
		\newline
		\newline\rule{0mm}{5ex}\textbf{Abgabedatum:} 
		\newline 14.02.2017
		\newline
	\end{tabular} 
\end{center}
\pagebreak

% ----------------------------------------------------------------------------------------------------------
% Verzeichnisse
% ----------------------------------------------------------------------------------------------------------
% TODO Typ vor Nummer
\renewcommand{\cfttabpresnum}{Tab. }
\renewcommand{\cftfigpresnum}{Abb. }
\settowidth{\cfttabnumwidth}{Abb. 10\quad}
\settowidth{\cftfignumwidth}{Abb. 10\quad}
\titlespacing{\section}{0pt}{12pt plus 4pt minus 2pt}{2pt plus 2pt minus 2pt}
\singlespacing
\rhead{INHALTSVERZEICHNIS}
\renewcommand{\contentsname}{II Inhaltsverzeichnis}
\phantomsection
\addcontentsline{toc}{section}{\texorpdfstring{II \hspace{0.35em}Inhaltsverzeichnis}{Inhaltsverzeichnis}}
\addtocounter{section}{1}
\tableofcontents
\pagebreak
\rhead{VERZEICHNISSE}


% ----------------------------------------------------------------------------------------------------------
% Inhalt
% ----------------------------------------------------------------------------------------------------------
% Abstände Überschrift
\titlespacing{\section}{0pt}{12pt plus 4pt minus 2pt}{-6pt plus 2pt minus 2pt}
\titlespacing{\subsection}{0pt}{12pt plus 4pt minus 2pt}{-6pt plus 2pt minus 2pt}
\titlespacing{\subsubsection}{0pt}{12pt plus 4pt minus 2pt}{-6pt plus 2pt minus 2pt}
% Kopfzeile
\renewcommand{\sectionmark}[1]{\markright{#1}}
\renewcommand{\subsectionmark}[1]{}
\renewcommand{\subsubsectionmark}[1]{}
\lhead{Kapitel \thesection}
\rhead{\rightmark}
\onehalfspacing
\renewcommand{\thesection}{\arabic{section}}
\renewcommand{\theHsection}{\arabic{section}}
\setcounter{section}{0}
\pagenumbering{arabic}
\setcounter{page}{1}

\newtheorem{bsp}{Beispiel}
\newtheorem{defnt}{Definition}

% ----------------------------------------------------------------------------------------------------------
% Einleitung
% ----------------------------------------------------------------------------------------------------------
\begin{abstract} 
	Die vorliegende Hausarbeit gibt einen Einstieg in die theoretische Fuzzy-Logik und behandelt dabei Fuzzy-Mengen, Fuzzy-Relation sowie Fuzzy-Operationen. Dem Leser soll bewusst gemacht werden worin sich die scharfe Mengenlehre von der der Fuzzy-Mengenlehre unterscheidet. An anschaulichen Beispielen wird erklärt was Linguistische Terme sind und wie diese repräsentiert werden können. Eigentliches Ziel ist die Untersuchung des Einsatzgebiets der Fuzzy-Regler, welches ein Teilgebiet der Fuzzy-Logik darstellt. Hier werden die Komponenten wie z.B. die Wissensbasis, das Fuzzyfizierungs- und Defuzzyfizierungs-Interface sowie die Entscheidungslogik vorgestellt. Da zur Zeit zwei Varianten (Mamdani sowie Takagi/Sugeno)  der Fuzzy-Regler verbreitet sind, werden diese beiden nacheinander vorgestellt und deren Vorteile aufgezeigt. Zuletzt werden die üblichen Defuzzifizierungsmethoden erklärt und an einem Beispiel veranschaulicht.
\end{abstract} 
\newpage

\section{Problemstellung}

Die gestellte Aufgabe zur praktischen Ausarbeitung kommt aus dem Themengebiet des autonomen Fahrens. Es soll ein Fuzzy Regler eingesetzt werden um ein Auto so zu steuern, dass es einem anderen, durch den Menschen gesteuerten, Auto folgt.  Hierbei ist zu beachten, dass das hier bei nur die Geschwindigkeit bzw. Bewegung in einer Achse kontrolliert wird. Das bedeutet, dass nur das Gas- sowie das Bremspedal jedoch nicht das Lenkrad geregelt werden muss. Außerdem gehen wir davon aus, dass es sich bei dem gesteuerten Auto um ein Auto mit Automatikgetriebe handelt und somit das Kuppeln sowie das Schalten der Gänge ebenfalls irrelevant ist. 

Bei der Implementierung sollte darauf geachtet werden, dass beide Autos, so weit wir möglich, den physikalischen Gesetzen unterliegen. Dies bedeutet, dass z.B. die Fahrwiderstände wie Luftwiderstand, Rollwiderstand usw. beachtet werden müssen. Ebenfalls sollten beide Autos durch eine physikalische Spezifikation, wie z.B. das Gewicht, definiert werden. Beide Autos sollten hierbei der gleichen physikalische Spezifikation unterliegen, um keinem der beiden Autos einen Vorteil zu bieten.


\begin{figure}
	\centering
	\includegraphics[width=0.9\linewidth]{img/practical/problem}
	\caption{Problemstellung}
	\label{fig:problem}
\end{figure}

\section{Implementierung}

Dieses Kapitel dient 

\subsection{Funktionale Programmiersprachen}



Für die Implementierung wird die Sprache \textbf{Scala} verwendet. Folgende Klassen werden verwendet.
\subsection{Modell des Autos}

\subsection{Konfiguration}

Mögliche Konfigurationen sind:

\begin{itemize}
	\item Partitionierung der Eingangsvariablen
	\item Partitionierung der Ausgangsvariablen
	\item Erstellung der linguistischen Regeln
	\item Wahl der Defuzzifizierungsmethode
\end{itemize}
Es wurden folgende Voreinstellungen für die Regelung gemacht:

\paragraph{Wertebreiche}
Distanz: 0-500 Meter \\
Geschwindigkeit: 0-250 km/h \\
Beschleunigung: -8000-4000 N \\

\paragraph{Partitionierung Distanz}
$\newline$
isVeryClose = $-\infty, 100, 200$ \\
isClose = $100, 200, 300$ \\
isNormal = $250, 300, 350$ \\
isFar = $300, 300, 500$ \\
isVeryFar = $400, 500, \infty$ \\

\paragraph{Partitionierung Geschwindigkeit}
$\newline$
isSlow = $-\infty, 15, 30$ \\
isFast = $15, 50, 100$ \\
isExtreme = $50, 150, 350$ \\

\paragraph{Partitionierung Beschleunigung}
$\newline$
fullBrake = $-\infty, -8000, -7000$ \\
medBrake = $-8000, -5000, -2000$ \\
brake = $-5000, -2000, 0$ \\
roll = $-500, 0, 500$ \\
speed = $0, 1000, 2000$ \\
medSpeed = $1000, 2000, 3000$ \\
fullSpeed = $2000, 3000, \infty$ \\

\paragraph{Regeln}

\begin{lstlisting}
IF Distance = isVeryFar AND Speed = isSlow THEN Force = fullSpeed
IF Distance = isVeryFar AND Speed = isFast THEN Force = fullSpeed
IF Distance = isVeryFar AND Speed = isExtreme THEN Force = fullSpeed
IF Distance = isFar AND Speed = isSlow THEN Force = fullSpeed
IF Distance = isFar AND Speed = isFast THEN Force = medSpeed
IF Distance = isFar AND Speed = isExtreme THEN Force = speed
IF Distance = isNormal AND Speed = isSlow THEN Force = roll
IF Distance = isNormal AND Speed = isFast THEN Force = roll
IF Distance = isNormal AND Speed = isExtreme THEN Force = roll
IF Distance = isClose AND Speed = isSlow THEN Force = brake
IF Distance = isClose AND Speed = isFast THEN Force = medBrake
IF Distance = isClose AND Speed = isExtreme THEN Force = fullBrake
IF Distance = isVeryClose AND Speed = isSlow THEN Force = fullBrake
IF Distance = isVeryClose AND Speed = isFast THEN Force = fullBrake
IF Distance = isVeryClose AND Speed = isExtreme THEN Force = fullBrake

\end{lstlisting}

\subsection{Fuzzy Controller}

\subsection{Grafische Oberfläche}


\section{Fazit}

\subsection{Probleme}

Leider ist dem Team zu spät aufgefallen, dass es sinnvoll gewesen wäre eine Ableitung des Abstands in die Regelung mit einzubringen. Die gesamte Regelung erscheint nun ein wenig Träge.

% ----------------------------------------------------------------------------------------------------------

\end{document}